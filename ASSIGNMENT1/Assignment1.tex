\documentclass[journal,12pt,twocolumn]{IEEEtran}
%
\usepackage{setspace}
\usepackage{gensymb}
%\doublespacing
\singlespacing

%\usepackage{graphicx}
%\usepackage{amssymb}
%\usepackage{relsize}
\usepackage[cmex10]{amsmath}
%\usepackage{amsthm}
%\interdisplaylinepenalty=2500
%\savesymbol{iint}
%\usepackage{txfonts}
%\restoresymbol{TXF}{iint}
%\usepackage{wasysym}
\usepackage{amsthm}
\usepackage{mathrsfs}
\usepackage{txfonts}
\usepackage{stfloats}
\usepackage{steinmetz}
\usepackage{bm}
\usepackage{cite}
\usepackage{cases}
\usepackage{subfig}
%\usepackage{xtab}
\usepackage{longtable}
\usepackage{multirow}
%\usepackage{algorithm}
%\usepackage{algpseudocode}
\usepackage{enumitem}
\usepackage{mathtools}
\usepackage{tikz}
\usepackage{circuitikz}
\usepackage{verbatim}
\usepackage{tfrupee}
\usepackage[breaklinks=true]{hyperref}
%\usepackage{stmaryrd}
\usepackage{tkz-euclide} % loads  TikZ and tkz-base
%\usetkzobj{all}
\usepackage{listings}
    \usepackage{color}                                            %%
    \usepackage{array}                                            %%
    \usepackage{longtable}                                        %%
    \usepackage{calc}                                             %%
    \usepackage{multirow}                                         %%
    \usepackage{hhline}                                           %%
    \usepackage{ifthen}                                           %%
  %optionally (for landscape tables embedded in another document): %%
    \usepackage{lscape}     
\usepackage{multicol}
\usepackage{chngcntr}
%\usepackage{enumerate}

%\usepackage{wasysym}
%\newcounter{MYtempeqncnt}
\DeclareMathOperator*{\Res}{Res}
%\renewcommand{\baselinestretch}{2}
\renewcommand\thesection{\arabic{section}}
\renewcommand\thesubsection{\thesection.\arabic{subsection}}
\renewcommand\thesubsubsection{\thesubsection.\arabic{subsubsection}}

\renewcommand\thesectiondis{\arabic{section}}
\renewcommand\thesubsectiondis{\thesectiondis.\arabic{subsection}}
\renewcommand\thesubsubsectiondis{\thesubsectiondis.\arabic{subsubsection}}

% correct bad hyphenation here
\hyphenation{op-tical net-works semi-conduc-tor}
\def\inputGnumericTable{}                                 %%

\lstset{
%language=C,
frame=single, 
breaklines=true,
columns=fullflexible
}
%\lstset{
%language=tex,
%frame=single, 
%breaklines=true
%}

\begin{document}
%


\newtheorem{theorem}{Theorem}[section]
\newtheorem{problem}{Problem}
\newtheorem{proposition}{Proposition}[section]
\newtheorem{lemma}{Lemma}[section]
\newtheorem{corollary}[theorem]{Corollary}
\newtheorem{example}{Example}[section]
\newtheorem{definition}[problem]{Definition}
%\newtheorem{thm}{Theorem}[section] 
%\newtheorem{defn}[thm]{Definition}
%\newtheorem{algorithm}{Algorithm}[section]
%\newtheorem{cor}{Corollary}
\newcommand{\BEQA}{\begin{eqnarray}}
\newcommand{\EEQA}{\end{eqnarray}}
\newcommand{\define}{\stackrel{\triangle}{=}}
\bibliographystyle{IEEEtran}
%\bibliographystyle{ieeetr}
\providecommand{\mbf}{\mathbf}
\providecommand{\pr}[1]{\ensuremath{\Pr\left(#1\right)}}
\providecommand{\qfunc}[1]{\ensuremath{Q\left(#1\right)}}
\providecommand{\sbrak}[1]{\ensuremath{{}\left[#1\right]}}
\providecommand{\lsbrak}[1]{\ensuremath{{}\left[#1\right.}}
\providecommand{\rsbrak}[1]{\ensuremath{{}\left.#1\right]}}
\providecommand{\brak}[1]{\ensuremath{\left(#1\right)}}
\providecommand{\lbrak}[1]{\ensuremath{\left(#1\right.}}
\providecommand{\rbrak}[1]{\ensuremath{\left.#1\right)}}
\providecommand{\cbrak}[1]{\ensuremath{\left\{#1\right\}}}
\providecommand{\lcbrak}[1]{\ensuremath{\left\{#1\right.}}
\providecommand{\rcbrak}[1]{\ensuremath{\left.#1\right\}}}
\theoremstyle{remark}
\newtheorem{rem}{Remark}
\newcommand{\sgn}{\mathop{\mathrm{sgn}}}
\providecommand{\abs}[1]{\left\vert#1\right\vert}
\providecommand{\res}[1]{\Res\displaylimits_{#1}} 
\providecommand{\norm}[1]{\left\lVert#1\right\rVert}
%\providecommand{\norm}[1]{\lVert#1\rVert}
\providecommand{\mtx}[1]{\mathbf{#1}}
\providecommand{\mean}[1]{E\left[ #1 \right]}
\providecommand{\fourier}{\overset{\mathcal{F}}{ \rightleftharpoons}}
%\providecommand{\hilbert}{\overset{\mathcal{H}}{ \rightleftharpoons}}
\providecommand{\system}{\overset{\mathcal{H}}{ \longleftrightarrow}}
	%\newcommand{\solution}[2]{\textbf{Solution:}{#1}}
\newcommand{\solution}{\noindent \textbf{Solution: }}
\newcommand{\cosec}{\,\text{cosec}\,}
\providecommand{\dec}[2]{\ensuremath{\overset{#1}{\underset{#2}{\gtrless}}}}
\newcommand{\myvec}[1]{\ensuremath{\begin{pmatrix}#1\end{pmatrix}}}
\newcommand{\mydet}[1]{\ensuremath{\begin{vmatrix}#1\end{vmatrix}}}
%\numberwithin{equation}{section}
\numberwithin{equation}{subsection}
%\numberwithin{problem}{section}
%\numberwithin{definition}{section}
\makeatletter
\@addtoreset{figure}{problem}
\makeatother
\let\StandardTheFigure\thefigure
\let\vec\mathbf
%\renewcommand{\thefigure}{\theproblem.\arabic{figure}}
\renewcommand{\thefigure}{\theproblem}
%\setlist[enumerate,1]{before=\renewcommand\theequation{\theenumi.\arabic{equation}}
%\counterwithin{equation}{enumi}
%\renewcommand{\theequation}{\arabic{subsection}.\arabic{equation}}
\def\putbox#1#2#3{\makebox[0in][l]{\makebox[#1][l]{}\raisebox{\baselineskip}[0in][0in]{\raisebox{#2}[0in][0in]{#3}}}}
     \def\rightbox#1{\makebox[0in][r]{#1}}
     \def\centbox#1{\makebox[0in]{#1}}
     \def\topbox#1{\raisebox{-\baselineskip}[0in][0in]{#1}}
     \def\midbox#1{\raisebox{-0.5\baselineskip}[0in][0in]{#1}}
\vspace{3cm}
\title{
SM5083 - BASICS OF PROGRAMMING
	}
\author{ RS Girish - EE20RESCH14005$^{*}$% <-this % stops a space
\thanks{*The author is with the Department
		of Electrical Engineering, Indian Institute of Technology, Hyderabad
		502285 India e-mail:  ee20resch14005@iith.ac.in. All content in this document is released under GNU GPL.  Free and open source.}
	}
\maketitle
\newpage
\tableofcontents
\bigskip
\renewcommand{\thefigure}{\theenumi}
\renewcommand{\thetable}{\theenumi}
\begin{abstract}
This paper contains solution to problem no 9(i) of Examples II Section of Analytical Geometry by Hukum Chand.
Links to Python codes are available below.  
\end{abstract}
Download python codes at 
\begin{lstlisting}
https://github.com/rsgirishkumar/SM5083/ASSIGNMENT1
\end{lstlisting}
\section{Problem}
Find the area of the quadrilateral formed by the points\\
\begin{align}
\begin{split}
\vec{A} = \myvec{1 \\ 1}, 
\vec{B} = \myvec{3 \\ 5}, 
\vec{C} = \myvec{-2 \\ 4}, 
\vec{D} = \myvec{-1 \\ -5}. 
\end{split}
\end{align}
\section{Solution}
Let the given points are indicated as below\\
\begin{align}
\begin{split}
\vec{A} = \myvec{1 \\ 1}, 
\vec{B} = \myvec{3 \\ 5}, 
\vec{C} = \myvec{-2 \\ 4}, 
\vec{D} = \myvec{-1 \\ -5}. 
\end{split}
\end{align}
\textbf{Step1}: Let us check whether the given points form a quadrilateral or not. This can be ascertained by collinearity check of any two sets of points i.e. 
either
($\vec{A}$, $\vec{B}$, $\vec{C}$) or ($\vec{A}$, $\vec{C}$, $\vec{D}$) or ($\vec{B}$, $\vec{C}$, $\vec{D}$) or ($\vec{A}$, $\vec{B}$, $\vec{D}$).
\\
\subsection{\textbf{Collinearity Check}}
Collinearity Check of points $\vec{A}$, $\vec{B}$, $\vec{D}$ i.e. 
\begin{align}
\begin{split}
\vec{A} = \myvec{1 \\ 1},\vec{B}= \myvec{3 \\ 5},\vec{D} = \myvec{-1 \\ -5}
\end{split}
\end{align}
\\
In Vector approach, If the rank of a matrix formed by the vectors of points
\myvec{x_1\\y_1}, \myvec{x_2\\y_2}, \myvec{x_3\\y_3} is \textbf{1 or 2} for any 3 X 3 matrix then the points are said to be \textbf{collinear}. If $\rho$\myvec{matrix}=3 then the points are non collinear.
\\
\Rightarrow   
\begin{equation}
\begin{math}$\rho$
\myvec{x_1 & y_1 & 1\\x_2 & y_2 & 1\\x_3 & y_3 & 1}
=3
\end{math}
\end{equation}
\\
Here the vectors are as below.\\
\begin{align}
    \vec{AB} = \brak{\vec{B} - \vec{A}}
    = \myvec{3\\5} - \myvec{1\\1}
    = \myvec{2\\4}
\\
    \vec{BD} = \brak{\vec{D} - \vec{B}}
    = \myvec{-1\\-5} - \myvec{3\\5}
    = \myvec{-4\\-10}
\\
    \vec{DA} = \brak{\vec{A} - \vec{D}}
    = \myvec{1\\1} - \myvec{-1\\-5}
    = \myvec{2\\6}
\end{align}
\\
The matrix formed by vectors is 
\begin{equation}
\begin{math}
\myvec{2 & 4 & 1\\-4 & -10 & 1\\2 & 6 & 1\\}
\\

Rank of the matrix: By reduction. 
\\
\Rightarrow
\myvec{2 & 4 & 1\\-4 & -10 & 1\\2 & 6 & 1\\}
\xleftrightarrow[]{R_2\leftrightarrow R_3}
\myvec{2 & 4 & 1\\2 & 6 & 1\\-4 & -10 & 1\\}
\xleftrightarrow[]{R_3\leftrightarrow R_1+R_2+R_3}
\myvec{2 & 4 & 1\\2 & 6 & 1\\0 & 0 & 3\\}
\xleftrightarrow[]{R_2\leftrightarrow R_2-R_1}
\myvec{2 & 4 & 1\\0 & 2 & 0\\0 & 0 & 3\\}
\end{math}
\end{equation}
The number of non-zero rows in the matrix = 3.
\\
\Rightarrow $\rho$\myvec{matrix} = 3.
\\
Here the AB, BD, DA are not collinear.
\\

Let us examine the lines generated by the given points in the Figure below:
\begin{figure}[!ht]
    \centering
    \includegraphics[width=\columnwidth]{QUAD.png}
    \caption{Quadrilateral ABCD}
    \label{fig:Quad ABCD}
\end{figure}
\\
The above figure clearly depicts the other set of points are not collinear. Hence the points given form a Quadrilateral.
\subsection{\textbf{Area of a Quadrilateral}}
\begin{center}
    $Area of \Box ABCD = Area of \triangle ACD + Area of \triangle ABC $
\end{center}
\\
$\because$ this does not fall into category of Square, Parallelogram or Rhombus, Trapezium.
\\
Area of a $\triangle ABC$ formed by points $\vec{A}=\myvec{x_1\\y_1},\vec{B}=\myvec{x_2\\y_2},\vec{C}=\myvec{x_3\\y_3}$ is given by
\begin{align}
\frac{1}{2}\mydet{1 & 1 & 1\\ \vec{A} & \vec{B} & \vec{C} }
\end{align}
\\
Area of $\triangle ABC$ formed by
$\vec{A}$ = \myvec{1\\1},$\vec{B}$=\myvec{3\\5},$\vec{C}$=\myvec{-2\\4} is given by
\begin{align}
\triangle ABC = 
\frac{1}{2}\mydet{
1 & 1 & 1\\1 & 3 & -2\\1 & 5 & 4\\}
= 9
\end{align}
\\
Area of $\triangle ACD$ formed by
$\vec{A}$=\myvec{1\\1},$\vec{C}$=\myvec{-2\\4},$\vec{D}$=\myvec{-1\\-5} is given by\\
\begin{align}
\triangle ACD = 
\frac{1}{2}\mydet{
1 & 1 & 1\\1 & -2 & -1\\1 & 4 & -5\\}
= 11.5
\end{align}
\\
\begin{lstlisting}
The area of quadrilateral ABCD = 9+11.5 = 20.5
\end{lstlisting}
\end{document}